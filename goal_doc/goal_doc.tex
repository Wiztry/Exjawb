\documentclass[a4paper]{article}
\usepackage[utf8]{inputenc}
\usepackage{amsmath}
\usepackage{url}
\usepackage{graphicx}
\usepackage{cite}


\newcommand{\signature}[2]{%
	\noindent%
	\textbf{{#1}}\\\\
	\begin{tabular}{@{}p{2.5in}@{}}
		\\ \hline \\[-.75\normalbaselineskip]
		\texttt{{#2}}
	\end{tabular} \hspace{0in}
	\begin{tabular}{@{}p{2.5in}@{}}
		\\ \hline \\[-.75\normalbaselineskip]
		Date
	\end{tabular}\\
}


\title{Master Thesis:\\\Large{Secure and stable comunication over wireless
		networks under heavy load}
\\\large{A Goal Document for a Master's Thesis work\\by}}
\author{Mattias Eklund \& Carl Nilsson Nyman}
\date{}

\begin{document}
\maketitle
\pagebreak

\begin{tabular}{lll}
	\textbf{Students:} & Mattias Eklund & Carl Nilsson Nyman \\
	\textbf{Civic reg nbrs:} & 19880116-0071& 19911008-1412\\
	\textbf{Email address:} & \texttt{mat10mek@student.lu.se}
	& \texttt{mat10cni@student.lu.se} \\
	\textbf{Main supervisor:} & TBD & \\
	\textbf{Examiner} & TBD & \\
	\textbf{Project start} & TBD & \\
	\textbf{Project end} & TBD & \\
	\textbf{Course Code} & EITM01 & 
\end{tabular}

%------------------------------------------------------------------------------%
%
%Ett förslag på examensarbete är att designa och implementera ett system för
%trådlös kommunikation mellan Uniti och perifera enheter. Primärt kommer detta
%att användas för kommunikaton mellan bilens infotainment-system och ett
%smartphone-gränssnitt, men kan även användas för att exempelvis låsa bilen med
%en separat enhet. Systemet ska även gärna vara skalbart till fler enheter,
%anpassat för kommunikation med Internet-of-Things (IoT) och Vehicle-to-Vehicle
%(V2V).
%
%Utvärdering och implementering av trådlösa protokoll för denna kommunikation är
%en stor del av arbetet, där även säkerhet är en viktig aspekt. Praktiskt kommer
%detta arbete med stor sannolikhet att integreras med övriga system i Unitis
%prototyp som ska visas upp i slutet av sommaren. För denna demo behöver Uniti
%en stabil och säker trådlös fjärrstyrning av bilen, samt en videolänk från de
%kameror som är monterade på bilen. Eftersom Uniti kör på Robot Operating System
%(ROS), behöver även en brygga implementeras mellan valt nätverksprotokoll och
%ROS. Användargränssnittet kommer att köras i Unity3D och kompatibilitet med
%detta behöver därmed säkerställas.
%
%Utöver detta visar vi gärna upp en lösning för trådlös inloggning och
%synkronisering av användarprofiler mellan en smartphone och bilens
%infotainment-system. Om det finns intresse och tid över så kan projektet enkelt
%utökas till att även inkludera design och implementering av ett
%användargränssnitt för tablets/smartphones i Unity3D.
%
%------------------------------------------------------------------------------%

\section{Introduction}
%Technology advancement in…

%Borde vara mer fokus på att informationsöverföring via trådlösa medier
%Fokus på stabilitet om något, onödigt att diskutera säkerhet för mycket ifall
%vi inte ska gå in på det.

%% 

%The need for secure and stable connection via wireless interfaces has become
%more and more relevant today. Users in a system should be able to trust a
%connections stability and security. By extension companies that supply services
%that depend on wireless connectivity should be able to guarantee stability
%and security in their connections.

When transmitting important data over a wireless connection it is important that
the data arrives in a timely manner. The medias used in a network should meet
all the requirements which the user imposes upon it. A dependency should be
enforced upon all the components in a network to ensure the stability of the
entire network.

%% De här bitarna bör snarare vara i introduktionen

%Wireless communication is allready used in the car industry today, and is used
%for (among other things) updating allready existing software on platforms
%allready in use. This is to ensure that bugs and security flaws in older models
%gets patched out as they are discovered.

%This communication is however not time critical as updates are performed when
%the vehicle is not in use. The communication that Uniti requires for this is
%much more time critical as input from a mobile device will affect the vehicle
%in very close to real time.

\section{Backgound and motivation}
%Recent research in this area…

%Bakgrunden borde vara mer en bakgrund till vårt exjobbs utförande...

%Uniti is a new company that is developing an electric vehicle which is supposed to drive autonomously. 
%For this purpose Uniti has expressed the need for a stable
%wireless communication link between their vehicle and peripheral devices, such
%as a tablet or laptop.

Uniti is a newly founded company that is developing an electric vehicle which will be driven autonomously.
In a couple of months Uniti will showcase their advancements in their development,
by hosting a demostrational event with their car in front of thousands of participants.
During this event the development of their car is not supposed to be finished, 
but rather supposed to be far along enough to showcase its electric-only propulsion and autonomous features. 

Today Uniti uses a UDP stream for communication between the vehicle and
peripheral devices, this is not really ideal as UDP-applications most likely
suffer from package-loss, errors, or duplication. This is fine for some
applications but for some applications it could be critical to miss out on certain information.

Uniti has expressed the need for a better solution for communicating with the vehicle.
This solution is supposed to let the vehicle be both monitored and in worst case controlled via outside devices.

%Using TCP will provide ordered and error-checked delivery of data, however, the
%application might have to wait for retransmission of packages. This approach is
%not ideal as latency might cause issues when controlling the vehicle.

%Furthermore, the media over which the communication is transmitted needs to be
%taken into consideration. Depending on the location, WiFi might not be
%available or stable enough. In such cases, could 4G be used instead? Or is 4G
%reliable enough to be used exclusively for this purpose? Reliable is in this
%case refering to low latency and low package loss.
%
\section{Project aims and challenges}
%The aim with this Master’s thesis work is to… To find a feasible solution to
%these questions are challenging…

%Det borde vara mer direkt på vad vi ska göra här och vad vill vi uppnå
%Vilka mål det finns att uppnå och dess begränsningar.

The goal of the thesis will is to evaluate a couple of different mediums of wireless network transmission to and from Uniti's vehicle, as well as evaluating a couple of different protocols for sending both video data and more discrete signals as in Start or Stop.

For our thesis we will be investigating the compatibility of setting up this connection both with 4G and WiFi Transmission. Furthermore we will also look into implementing both RTP and MMT for sending video data.

Uniti has in turn set the requirements on our solution that it is supposed to work in an environment with at least 10000 people. Our transmission of discrete signals has to be guaranteed to arrive with a worst-case latency of 50ms. Finally video transmission should work with an average latency of around 100ms.
 

%The goal of this thesis is to evaluate different mediums of wireless network
%transmission to and from a unit in motion. This includes implementing a wireless
%connection between the unit and different peripheral devices. This work involves
%certain research into different network protocols and wireless technologies such
%as 4G or WiFi.

%The focus of the work will be to find a solution which results in a long range
%transmission with minimal latency and a minimal rate of packet-loss.

%The aim with this Master's thesis work is to research and implement a stable
%connection between Uniti's car and peripheral devices such as a telephone or
%laptop via wireless connection such as WiFi, 4G, or Bluetooth.

%This connection should be able act with an acceptable latency and with a minimal
%rate of packet-loss.

%The goal within this work is to research and conclude what the best media (4G,
%Wifi etc.) and the best format (network protocol etc.)


%The problems lies in ensuring that the connection remains stable on
%networks that are under heavy load while not contributing too much to the
%latency of the connection. Setting up the connection and making it secure 
%is also of great concern but our aim within this project has its focus 
%set on ensuring low latency and stability between units.

\section{Approach and methodology}
%The thesis project will be based on…
%Hur ska vi göra exjobbet och i vilka steg tar vi det?

%Detta borde snarare handla om hur vi först gör detta och sedan gör detta.

The majority of our thesis project will be put into implementing a wireless
connection between Uniti's vehicle and a test platform which will help us
evaluate our thesis.

The first part of our thesis project will be put into implementing a connection
with the platform which their vehicle uses (Robot operating system), which will
be implemented through the use of the package rosbridge.

Beyond this we will be able to implement a connection from the vehicle to a test
platform through more conventional means.

Finally we will be able to substitute the different factors in our connection,
such as medium and protocol, from the vehicle to the test platform with
different mediums and protocols to test our thesis and in that achieve the goal
we set.

%The thesis project will be based on the requirements set by the different parts
%of the company Uniti's infrastructural software. This software is based
%partially in their car which uses Robot Operating System (ROS) but also in
%their cell phone application which is built in Unity3D. Most of this project
%will include implementing a network link between these parts. Uniti has
%already started looking at developing some kind of link, which is what
%we will initially work with to build upon.


\section{Previous work}
%Back-ground knowledge and experiences can be found among the professors and PhD
%students at the department…

%Tidigare arbete inom området ska mer röra information vi behöver utifrån för
%att göra vårt exjobb Detta är det sista man skriver i rapporten och man behöver
%nog inte läsa artiklar djupare än att veta vad man kan ta ifrån dem.
%Det är här t.ex. orimligt att använda Wi-Fi för att tänk på vad som händer när
%man kör ifrån en accesspunkt

%We have read an article \cite{NIST_report} concerning more general
%internet security on both the link layer and application layer and
%it have broght up some interesting points concerning both security and
%stability. Mostly concerning what aspects you have to consider when specifying a
%network group.

%According to Giuseppe's analysis of IEEE 802.11 performance
%\cite{PAIEEE_article}, the performance is, not very suprisingly, affected by the
%number of units on the network. This will be an issue during showcasing of
%Uniti's vehicle as low latency is a requirement for safe remote operation of the
%vehicle as during showcasing the network will most likely be heavily populated
%by devices.

%These articles went into some detail about the 802.11 standard for Wifi.
%We intend to start our thesis project by testing to use 802.11 for our 
%wireless communication setup.


\section{Advancements and Outcome}
%Verification of the theoretical knowledge will be shown by real-world
%experiments… A scientific conference paper can be expected…

%Vad har vi åstadkommit i vårt arbete, vad är det tekniska eller
%forskningsrelaterade vi har tagit fram som inte tidigare fanns?


% --- DRAFT FOR ADVANCEMENTS AND OUTCOME --- %

The outcome from this thesis work is made up of two parts, one practical and one
theoretical.

The practical part consists of the connection implementation between Uniti's
vehicle and the peripheral devices that will be based on the theoretical
findings.

The theoretical part consists of the test-results from the different
implementations of the connection between Uniti's vehicle and peripheral
devices. The knowledge obtained from these tests could be used as a basis for
decisions regarding implementation of wireless application and which protocol
and connection media that should be used.

% ---------------- END DRAFT --------------- %

The theoretical knowledge used within this project will be rather easy
to verify the success of, seeing as the implementation of the different 
protocols and integration with different wireless medias will provide
feedback easily. Something that is harder to verify by oneself is whether
or not a certain solution is safe in terms of network security or whether
or not it is stable under heavy loads, as these things require larger
experiments to verify.

\section{Resources}
%Linux based workstations, placed in thesis workers rooms, will be
%available for simulation and implementation.

%Vilka resurser finns tillgängliga till oss när vi gör vårt exjobb.


We will work with our thesis at Uniti's office in Lund where Uniti will provide
us with the resources necessary to complete this thesis work. Below is a list of
agreed upon resources that will be provided.

\begin{itemize}
	\item[Workstations] Stationary computers to work on.
	\item[Laptop] Just like a stationary computer this will be used to produce
		code on.
	\item[Test platforms] There will be two test platforms available, one full
		size and one small scale. The small scale test platform will run the same
		ROS core on weaker hardware.
	\item[TOBY-L210] Long range radio module. 4G with 3G/2G fallback.
	\item[EMMY-w163] Short range radio module. WiFi.
	\item[EVK-L2x] Evaluation kit for TOBY-L2x
	\item[EVK-EMMY-W1] Evaluation kit for EMMY-w1x
	\item[source code] Current source code.

\end{itemize}

During our thesis work, we will work by integrating our ideas into Uniti's
current source code.
Network tests and simulations will be performed with the help of Uniti's
equipment.



\bibliography{goal_doc.bib}{}
\bibliographystyle{plain}


This Goal Document is approved by:

\signature{Main supervisor}{<name here>}\\

\signature{Examiner}{<name here>}

\end{document}
