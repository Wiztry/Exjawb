\documentclass[a4paper]{article}
\usepackage[utf8]{inputenc}
\usepackage{amsmath}
\usepackage{url}
\usepackage{graphicx}
\usepackage{cite}


\newcommand{\signature}[2]{%
	\noindent%
	\textbf{{#1}}\\\\
	\begin{tabular}{@{}p{2.5in}@{}}
		\\ \hline \\[-.75\normalbaselineskip]
		\texttt{{#2}}
	\end{tabular} \hspace{0in}
	\begin{tabular}{@{}p{2.5in}@{}}
		\\ \hline \\[-.75\normalbaselineskip]
		Date
	\end{tabular}\\
}


\title{Cool title\\\large{A Goal Document for a Master's Thesis work\\by}}
\author{Mattias Eklund \& Carl Nilsson Nyman}
\date{}

\begin{document}
\maketitle
\pagebreak

\begin{tabular}{lll}
	\textbf{Students:} & Mattias Eklund & Carl Nilsson Nyman \\
	\textbf{Civic reg nbrs:} & 19880116-0071& 19911008-1412\\
	\textbf{Email address:} & \texttt{mat10mek@student.lu.se}
	& \texttt{mat10cni@student.lu.se} \\
	\textbf{Main supervisor:} & & \\
	\textbf{Examiner} & & \\
	\textbf{Project start} & & \\
	\textbf{Project end} & & \\
	\textbf{Course Code} & & 
\end{tabular}

%------------------------------------------------------------------------------%
%
%Ett förslag på examensarbete är att designa och implementera ett system för
%trådlös kommunikation mellan Uniti och perifera enheter. Primärt kommer detta
%att användas för kommunikaton mellan bilens infotainment-system och ett
%smartphone-gränssnitt, men kan även användas för att exempelvis låsa bilen med
%en separat enhet. Systemet ska även gärna vara skalbart till fler enheter,
%anpassat för kommunikation med Internet-of-Things (IoT) och Vehicle-to-Vehicle
%(V2V).
%
%Utvärdering och implementering av trådlösa protokoll för denna kommunikation är
%en stor del av arbetet, där även säkerhet är en viktig aspekt. Praktiskt kommer
%detta arbete med stor sannolikhet att integreras med övriga system i Unitis
%prototyp som ska visas upp i slutet av sommaren. För denna demo behöver Uniti
%en stabil och säker trådlös fjärrstyrning av bilen, samt en videolänk från de
%kameror som är monterade på bilen. Eftersom Uniti kör på Robot Operating System
%(ROS), behöver även en brygga implementeras mellan valt nätverksprotokoll och
%ROS. Användargränssnittet kommer att köras i Unity3D och kompatibilitet med
%detta behöver därmed säkerställas.
%
%Utöver detta visar vi gärna upp en lösning för trådlös inloggning och
%synkronisering av användarprofiler mellan en smartphone och bilens
%infotainment-system. Om det finns intresse och tid över så kan projektet enkelt
%utökas till att även inkludera design och implementering av ett
%användargränssnitt för tablets/smartphones i Unity3D.
%
%------------------------------------------------------------------------------%

\section{Introduction}
%Technology advancement in…
The need for secure and stable connections via wireless interfaces has become
more and more relevant today.  Users in a system should be able to trust a
connections stability and security. By extension companies that supply services
that depend on wireless connectivity should be able to guarantee stability
and security in their connections.


\section{Backgound and motivation}
%Recent research in this area…

We have read some articles\cite{NIST_report} concerning more general
internet security on both the link layer and application layer and
they have broght up som interesting points concerning different holes
in security. Alot of research have lately concerned preventing 
DoS (Denial-of-Service) attacks while there also has been always been a
steady research into the security holes that wireless networks give rise to. 


\section{Project aims and challenges}
%The aim with this Master’s thesis work is to… To find a feasible solution to
%these questions is challenging…

The aim with this Master's thesis work is to reseach and implement a stable and
secure connection between Uniti's car and mobile devices such as a telephone or
laptop via wireless connection on networks.

The problems lies in ensuring that the connection remains stable and secure on
networks that are under heavy load while not contributing too much to the
latency of the connection.

\section{Approach and methodology}
%The thesis project will be based on…
The thesis project will be based on the requirements set by the different parts
of the company Uniti's infrastructural software. This software is based
partially in their car which uses Robot Operating System (RoG) but also in
their cell phone application which is built in Unity3D. Most of this project
will include implementing a network link between these parts. Uniti has
already started looking at developing some kind of link, which is what
we will initially work with to build upon.


\section{Previous work}
%Back-ground knowledge and experiences can be found among the professors and PhD
%students at the department…



\section{Advancements and Outcome}
%Verification of the theoretical knowledge will be shown by real-world
%experiments… A scientific conference paper can be expected…
The theoretical knowledge used within this project will be rather easy
to verify the success of, seeing as the implementation of the different 
protocols and integration with different wireless medias will provide
feedback easily. Something that is harder to verify by oneself is whether
or not a certain solution is safe in terms of network security or whether
or not it is stable under heavy loads, as these things require larger
experiments to verify.

\section{Resources}
%PC or Linux based workstations, placed in thesis workers rooms, will be
%available for simulation and implementation. 
We will work with our thesis at Uniti's office in lund, where we will
be placed in some kind office room. Uniti will also provide us with
workstations which we can work upon. During the thesis we will 
implement certain software which we will integrate with Uniti's
own source code. Network tests and simulation will also be performed
on Uniti's equipment.


\bibliography{goal_doc.bib}{}
\bibliographystyle{plain}


This Goal Document is approved by:

\signature{Main supervisor}{<name here>}\\

\signature{Examiner}{<name here>}

\end{document}
