\documentclass[a4paper]{article}
\usepackage[utf8]{inputenc}
\usepackage{amsmath}
\usepackage{url}
\usepackage{graphicx}



\newcommand{\signature}[2]{%
	\noindent%
	\textbf{{#1}}\\\\
	\begin{tabular}{@{}p{2.5in}@{}}
		\\ \hline \\[-.75\normalbaselineskip]
		\texttt{{#2}}
	\end{tabular} \hspace{0in}
	\begin{tabular}{@{}p{2.5in}@{}}
		\\ \hline \\[-.75\normalbaselineskip]
		Date
	\end{tabular}\\
}


\title{Cool title\\\large{A Goal Document for a Master's Thesis work\\by}}
\author{Mattias Eklund \& Carl Nilsson Nyman}
\date{}

\begin{document}
\maketitle
\pagebreak

\begin{tabular}{lll}
	\textbf{Students:} & Mattias Eklund &Carl Nilsson Nyman \\
	\textbf{Civic reg nbrs:} & 19880116-0071& 19911008-1412\\
	\textbf{Email address:} & \texttt{mat10mek@student.lu.se}
	& \texttt{mat10cni@student.lu.se} \\
	\textbf{Main supervisor:} & & \\
	\textbf{Examiner} & & \\
	\textbf{Project start} & & \\
	\textbf{Project end} & & \\
	\textbf{Course Code} & & 
\end{tabular}

%------------------------------------------------------------------------------%
%
%Ett förslag på examensarbete är att designa och implementera ett system för
%trådlös kommunikation mellan Uniti och perifera enheter. Primärt kommer detta
%att användas för kommunikaton mellan bilens infotainment-system och ett
%smartphone-gränssnitt, men kan även användas för att exempelvis låsa bilen med
%en separat enhet. Systemet ska även gärna vara skalbart till fler enheter,
%anpassat för kommunikation med Internet-of-Things (IoT) och Vehicle-to-Vehicle
%(V2V).
%
%Utvärdering och implementering av trådlösa protokoll för denna kommunikation är
%en stor del av arbetet, där även säkerhet är en viktig aspekt. Praktiskt kommer
%detta arbete med stor sannolikhet att integreras med övriga system i Unitis
%prototyp som ska visas upp i slutet av sommaren. För denna demo behöver Uniti
%en stabil och säker trådlös fjärrstyrning av bilen, samt en videolänk från de
%kameror som är monterade på bilen. Eftersom Uniti kör på Robot Operating System
%(ROS), behöver även en brygga implementeras mellan valt nätverksprotokoll och
%ROS. Användargränssnittet kommer att köras i Unity3D och kompatibilitet med
%detta behöver därmed säkerställas.
%
%Utöver detta visar vi gärna upp en lösning för trådlös inloggning och
%synkronisering av användarprofiler mellan en smartphone och bilens
%infotainment-system. Om det finns intresse och tid över så kan projektet enkelt
%utökas till att även inkludera design och implementering av ett
%användargränssnitt för tablets/smartphones i Unity3D.
%
%------------------------------------------------------------------------------%
\section{Introduction}
%Technology advancement in…
The need for secure and stable connections via wireless interfaces has become
more and more relevant today. Users of a system should be able to trust and
depend on a connections stability and security, and by extension, companies that
supply services that depend on wireless connectivity have to be able to
guarantie the stability and security of their connections.


\section{Backgound and motivation}
%Recent research in this area…




\section{Project aims and challenges}
%The aim with this Master’s thesis work is to… To find a feasible solution to
%these questions is challenging…

The aim with this Master's thesis work is to reseach and implement a stable and
secure connection between Uniti's car and mobile devices such as a telephone or
laptop via wireless connection on networks.

The problems lies in ensuring that the connection remains stable and secure on
networks that are under heavy load while not contributing too much to the
latency of the connection.

\section{Approach and methodology}
%The thesis project will be based on…



\section{Previous work}
%Back-ground knowledge and experiences can be found among the professors and PhD
%students at the department…


\section{Advancements and Outcome}
%Verification of the theoretical knowledge will be shown by real-world
%experiments… A scientific conference paper can be expected…


\section{Resources}
%PC or Linux based workstations, placed in thesis workers rooms, will be
%available for simulation and implementation. 


This Goal Document is approved by:

\signature{Main supervisor}{<name here>}\\

\signature{Examiner}{<name here>}

\end{document}
